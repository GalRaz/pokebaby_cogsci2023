% Template for Cogsci submission with R Markdown

% Stuff changed from original Markdown PLOS Template
\documentclass[10pt, letterpaper]{article}

\usepackage{cogsci}
\usepackage{pslatex}
\usepackage{float}
\usepackage{caption}

% amsmath package, useful for mathematical formulas
\usepackage{amsmath}

% amssymb package, useful for mathematical symbols
\usepackage{amssymb}

% hyperref package, useful for hyperlinks
\usepackage{hyperref}

% graphicx package, useful for including eps and pdf graphics
% include graphics with the command \includegraphics
\usepackage{graphicx}

% Sweave(-like)
\usepackage{fancyvrb}
\DefineVerbatimEnvironment{Sinput}{Verbatim}{fontshape=sl}
\DefineVerbatimEnvironment{Soutput}{Verbatim}{}
\DefineVerbatimEnvironment{Scode}{Verbatim}{fontshape=sl}
\newenvironment{Schunk}{}{}
\DefineVerbatimEnvironment{Code}{Verbatim}{}
\DefineVerbatimEnvironment{CodeInput}{Verbatim}{fontshape=sl}
\DefineVerbatimEnvironment{CodeOutput}{Verbatim}{}
\newenvironment{CodeChunk}{}{}

% cite package, to clean up citations in the main text. Do not remove.
\usepackage{apacite}

% KM added 1/4/18 to allow control of blind submission
\cogscifinalcopy

\usepackage{color}

% Use doublespacing - comment out for single spacing
%\usepackage{setspace}
%\doublespacing


% % Text layout
% \topmargin 0.0cm
% \oddsidemargin 0.5cm
% \evensidemargin 0.5cm
% \textwidth 16cm
% \textheight 21cm

\title{Familiarity preference something something???}


\author{Gal Raz$^1$ (galraz@mit.edu), \bf{Anjie Cao$^2$  (anjiecao@stanford.edu)},\\ \bf{Michael C. Frank$^2$ (mcfrank@stanford.edu)},
 and \bf{Rebecca Saxe$^1$ (saxe@mit.edu)} \\
$^1$Department of Brain and Cognitive Sciences, MIT, $^2$Department of Psychology, Stanford University \\ }


\begin{document}

\maketitle

\begin{abstract}
haha

\textbf{Keywords:}
decision making; learning; bayesian modeling; cognitive development
\end{abstract}

\hypertarget{introduction}{%
\section{Introduction}\label{introduction}}

\hypertarget{experiment-1}{%
\section{Experiment 1}\label{experiment-1}}

\hypertarget{methods}{%
\subsection{Methods}\label{methods}}

\hypertarget{participants}{%
\subsection{Participants}\label{participants}}

66 children completed a task modified from the adult self-paced looking
time studies reported in CITE. Following our pre-registration (LINK), 2
children were excluded from the analysis because their performance in
the attention-check task failed to meet the inclusion criteria. We also
excluded trials with looking time that were three absolute deviations
away from the median in the log-transformed space across participants.
The final datasets includes 64 children in total (3YO: N = 18; 4YO: N =
26; 5YO: N = 20). All participants were recruited in a
university-affiliated research preschool.

\hypertarget{stimuli}{%
\subsection{Stimuli}\label{stimuli}}

We used a subset of stimuli created for the adult self-paced looking
time studies. In the previous study, we created a set of animated
creatures using Spore (a game developed by Maxis in 2008). Half of the
creatures had high perceptual complexity, and half had low perceptual
complexity. We used the high perceptual complexity stimuli for the
current study.

\hypertarget{procedure}{%
\subsection{Procedure}\label{procedure}}

Children were tested individually in a test room by an experimenter. The
experimenter invited the child to ``meet some monster friends'' and then
familiarized the child with the laptop computer used to present the
experiment. Before the test, each child went through a practice phase
where they practiced pressing the space bar to move on to the next
trial. The child was instructed that they can press the key and move on
to meet more monster friends whenever they want.

On each trial, the child would see an animated creature appear on the
screen. The child can move on to the next trial by pressing the space
bar. Each block consisted of six trials. Usually, the same creature will
be shown repeatedly (the background stimulus), but each block could
contain either zero or one deviant trial. Deviant trials were trials
that present a different creature from the background stimulus. Deviant
trials appeared on the second, the fourth, or the sixth trial of the
block. Each child saw eight blocks in total.

At the offset of each block, a memory task was presented to ensure
children are appropriately attending to the task. The memory task was a
2-Alternative Forced Choice (2AFC) question, asking the children to
identify which of the two stimuli they have seen before. The pair of
stimuli contained one stimulus used as a background stimulus in the
preceding block and a novel stimulus that did not appear anywhere else
in the experiment.

\hypertarget{results-and-discussion}{%
\subsection{Results and discussion}\label{results-and-discussion}}

We anticipated that the preschooler children would show patterns of
habituation and dishabituation similar to adults. We also expected to
see developmental changes in the shape of habituation trajectories. Our
pre-registered mixed-effect mod includes a three-way interaction term
between age (in months), trial number, and trial type (background or
deviant) to predict log-transformed looking time. We only found a main
effect of trial number, suggesting that participants look shorter at
latter trials (\(\beta\) = -0.1, \emph{SE} = 0.05, \emph{t} = -2.05,
\emph{p} = 0.04). Since there was no reliable age effect, we deviated
from the analysis plan and ran a mixed-effect model only including
two-way interaction between trial number and trial type. In this model,
all predictors were significant (all \emph{p} \textless{} 0.01),
suggesting our paradigms successfully captured habituation and
dishabituation in preschoolers.

We also explored the potential familiarity preference by comparing the
looking time at the second background trial and the second deviant
trial. Under the Hunter \& Ames (1988), the second trial in each block
is most likely to yield a familiarity preference, since participants
receive the least amount of familiarization with the background stimulus
in a block. If there is a familiarity preference, participants should
look longer at a background trial than a deviant trial. However, we did
not find evidence supporting this prediction. We ran a mixed effect
model predicting looking time at the second trial with trial type as the
predictor. There was a significant trial type effect in the opposite
direction, suggesting participants looked longer at the deviant trial
than the background trial even with as little as one trial of
familiarization time (\(\beta\) = 0.41, \emph{SE} = 0.03, \emph{t} =
12.24, \emph{p} \textless{} 0.01).

In summary, This current experiment replicated the finding in We
captured habituation and dishabituation with a developmental sample.
More importantly, under the current paradigm, we did not find any
evidence of familiarity preference for this younger age group. We moved
to the infant samples in the next experiment.

\hypertarget{experiment-2}{%
\section{Experiment 2}\label{experiment-2}}

\hypertarget{methods-1}{%
\subsection{Methods}\label{methods-1}}

Classical theories of attentional preferences in development posit that
younger participants are more likely to exhibit familiarity preferences
for a given stimulus due to their reduced encoding speed (CITE). Indeed,
prior studies show a reversal from novelty to familiarity preferences
when testing younger samples (CITE that grammar infant study, Cyr, H\&A
?). To this end, in Experiment 2, we test the model predictions in
preverbal infants, and try to evoke both familiarity and novelty
preferences by interrupting familiarization to a stimulus at different
time points.

To this end, we developed a new infant looking paradigm which mirrors
the adult (Cao et al., 2022) and preschooler paradigm (Experiment 1). In
this paradigm, infants are familiarized to multiple stimuli for
different exposure durations within a single session in a blocked
design. This is in contrast to the standard infant
familiarization/habituation paradigm in which typically infants are
familiarized to only one stimulus throughout the experiment, which makes
the effect of exposure duration difficult to estimate. By presenting
infants with multiple blocks with varying familiarization times, we can
now measure the effect of familiarization on preference within subjects.

To get a dense estimate of exposure durations, we pre-registered and ran
two experiments, sequentially, with two sets of exposure durations. The
first experiment showed infants blocks of 0, 4 or 8 familiarizations
(Exp A; pre-registered here {[}link{]}). The second experiment showed
infants blocks of 1, 3 or 9 familiarizations (Exp B; pre-registered here
{[}link{]}).

\hypertarget{participants-1}{%
\subsection{Participants}\label{participants-1}}

We tested a combined sample of 66 7-10 month old infants, with 31 in Exp
A and 35 in Exp B (\(M_{age}\) = 9.5 months, 31 female). 5 participants
were excluded completely due to fussiness (Exp A:, 2, Exp B: 3). We also
excluded an additional 72 individual test trials in which 1) infants
looked at the stimuli for less than a total 2 seconds, 2) there were
momentary external distractions in the home of the infant or 3) the gaze
classifier (see \textbf{Looking time coding}) had an average
classification confidence of less than 50\%. Data collection was
performed synchronously on Zoom, and infants were recruited from Lookit
and Facebook.

\hypertarget{stimuli-1}{%
\subsection{Stimuli}\label{stimuli-1}}

Infants saw a different stimulus set from the preschoolers. In a
previous study, we showed the infants the Spore stimulus set used in
preschoolers, but we were unable to elicit robust dishabituation
effects, which we considered critical for the validity of the paradigm.
Instead, we presented with a series of animated animals, which we
created using Unity assets (\url{https://tinyurl.com/469xxrn7}). The
animals were walking, crawling or swimming, depending on the species.

\hypertarget{procedure-1}{%
\subsection{Procedure}\label{procedure-1}}

This experiment followed a block structure, where each block was divided
into two sections: 1) A familiarization period and 2) a test event. Each
block was preceded by our lab-standard attention getter, a salient
rotating star. During the familiarization period, the infant was
familiarized to a particular animal, the background, in a series of
familiarization trials. Each familiarization trial was a 5 second long
presentation of the animal, during which the animated animal appears
behind curtains that open for 1 second, then the animal was presented in
that size for 3 seconds, and then the curtains closed again for 1
second. We refer to the number of times the curtain opened and closed as
the ``familiarization duration'', which varied between blocks.

During the test event, the infant saw either the same background animal
again, or a novel animal, the deviant. The onset of the test event was
not marked by any visual markers, but a bell sound is played as the
curtains open, to maximize the chance of engagement during the test
trial. The test event used an infant-controlled procedure, in which the
experimenter terminated the trial when the infant looked away for more
than three consecutive seconds. Looking time was defined as the total
time that the infant spends looking at the screen from the onset of the
stimulus until the first two consecutive seconds of the infant looking
away from the screen. The discrepancy between the experimenter criterion
and the looking time criterion was to be conservative in stopping trials
to avoid early trial terminations. If the infant did not meet the
lookaway criterion after 60 seconds of being presented with the test
event, the next block automatically began and infants' looking time for
this test event was recorded as 60 seconds.

Each baby saw six blocks: Three different familiarization durations (0,
4 and 8 in Exp A, and 1, 3 and 9 in Exp B) appeared twice each, once for
each test event type (background or deviant).

\hypertarget{looking-time-coding}{%
\subsection{Looking time coding}\label{looking-time-coding}}

To code the infants' gaze we used iCatcher+, a validated tool developed
for robust and automatic annotation of infants' gaze direction from
video (Erel et al., 2022). To obtain trial-wise looking times, we merged
iCatcher+ annotations with trial timing information, thereby fully
replacing the field-standard of manual coding of looking times.

\hypertarget{results-and-discussion-1}{%
\subsection{Results and discussion}\label{results-and-discussion-1}}

\hypertarget{cache-pre-registered-models}{%
\section{cache pre-registered
models}\label{cache-pre-registered-models}}

To test the prediction that partial encoding elicits familiarity
preferences, while complete encoding elicits novelty preferences, we
pre-registered a model which allows for a non-linear interaction between
exposure duration by adding a quadratic effect of familiarization
duration, and its interaction with novelty. We found that neither the
main effect, nor the interaction of that quadratic term were significant
(Model 1a; main effect \& interaction of poly2 terms), while the
interaction of novelty with the linear term was significant (Model 1a;
interaction of poly1 terms). This suggests that novelty preferences get
stronger as a function familiarization duration, but that there is no
special effect of partial encoding as posited by H\&A. Furthermore,
there was a significant decrease in looking times to the familiar items
as a function of familiarization duration (Model 1a, poly(fam\_duration,
1)), indicating that infants habituated to familiar stimuli in our
paradigm. We next tested specifically for the existence of familiarity
preference in our dataset. After finding a hint of a familiarity
preference after four familiarizations in the first study, which did not
turn out significant in an exploratory analysis (Model 2a: test type
effect), we argued that if familiarity preferences are driven by partial
encoding, any condition in which there were fewer exposures should also
reveal a familiarity preference. We therefore tested whether looking to
the familiar stimulus was longer in all trial with four or less
exposures, which it did not (Model 2b; test type effect). Novelty
preferences on the other hand were robust after 8 (Model 3a; test type
effect) and 9 familiarizations (Model 3b; test type effect), as well as
in the combined dataset (Model 3c; test type effect).

\hypertarget{general-discussion}{%
\section{General discussion}\label{general-discussion}}

\hypertarget{references}{%
\section{References}\label{references}}

\setlength{\parindent}{-0.1in} 
\setlength{\leftskip}{0.125in}

\noindent

\bibliographystyle{apacite}


\end{document}
