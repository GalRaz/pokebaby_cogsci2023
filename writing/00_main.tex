% Template for Cogsci submission with R Markdown

% Stuff changed from original Markdown PLOS Template
\documentclass[10pt, letterpaper]{article}

\usepackage{cogsci}
\usepackage{pslatex}
\usepackage{float}
\usepackage{caption}

% amsmath package, useful for mathematical formulas
\usepackage{amsmath}

% amssymb package, useful for mathematical symbols
\usepackage{amssymb}

% hyperref package, useful for hyperlinks
\usepackage{hyperref}

% graphicx package, useful for including eps and pdf graphics
% include graphics with the command \includegraphics
\usepackage{graphicx}

% Sweave(-like)
\usepackage{fancyvrb}
\DefineVerbatimEnvironment{Sinput}{Verbatim}{fontshape=sl}
\DefineVerbatimEnvironment{Soutput}{Verbatim}{}
\DefineVerbatimEnvironment{Scode}{Verbatim}{fontshape=sl}
\newenvironment{Schunk}{}{}
\DefineVerbatimEnvironment{Code}{Verbatim}{}
\DefineVerbatimEnvironment{CodeInput}{Verbatim}{fontshape=sl}
\DefineVerbatimEnvironment{CodeOutput}{Verbatim}{}
\newenvironment{CodeChunk}{}{}

% cite package, to clean up citations in the main text. Do not remove.
\usepackage{apacite}

% KM added 1/4/18 to allow control of blind submission
\cogscifinalcopy

\usepackage{color}

% Use doublespacing - comment out for single spacing
%\usepackage{setspace}
%\doublespacing


% % Text layout
% \topmargin 0.0cm
% \oddsidemargin 0.5cm
% \evensidemargin 0.5cm
% \textwidth 16cm
% \textheight 21cm

\title{No evidence for familiarity preferences after partial exposure to
visual concepts in preschoolers and infants}


\author{Gal Raz$^1$ (galraz@mit.edu), \bf{Anjie Cao$^2$  (anjiecao@stanford.edu)},\\ \bf{Ming Khong Bui$^3$  (mbui100@csu.fullerton.edu)},\\ \bf{Michael C. Frank$^2$ (mcfrank@stanford.edu)},
 and \bf{Rebecca Saxe$^1$ (saxe@mit.edu)} \\
$^1$Department of Brain and Cognitive Sciences, MIT, $^2$Department of Psychology, Stanford University, \\ $^3$Department of Communicative Sciences and Disorders, California State University \\
}

\newlength{\cslhangindent}
\setlength{\cslhangindent}{1.5em}
\newenvironment{CSLReferences}%
  {}%
  {\par}

\begin{document}

\maketitle

\begin{abstract}


\textbf{Keywords:}
developmental psychology; familiarity prefernces; habituation; learning;
attention; cognitive development
\end{abstract}

\hypertarget{introduction}{%
\section{Introduction}\label{introduction}}

Throughout development, humans are inundated with visual information.
Infants and young children constantly decide how much time to spend
looking at what is in front of them and when to move on to something
else {[}Haith (1980); dweck2017needs; raz2020learning{]}. Developmental
psychologists have long relied on infants' ability to decide what to
look at, when making inferences about infants' mental representations
{[}Aslin (2007); baillargeon1985object; fantz1963pattern{]}. In a
typical study measuring looking time, infants are presented with the
same stimulus repeatedly until their looking time decreases
(i.e.~habituation). Then, they are presented with a new stimulus, and
the change in their looking time is used as evidence for cognitive
capacities. Despite extensive use of looking time as a measure, the
rules underlying infants' decision to keep looking or look away are not
well understood. In this paper, we conduct a direct empirical test of
the relationships between prior exposure and looking time to familiar
and novel stimuli.

One dominant framework for infant looking is that the dynamics of
looking time are governed by the dynamics of learning (Hunter \& Ames,
1988). This framework has been used to derive qualitative predictions
about looking time as a function of prior exposure and stimulus
complexity. If infants have sufficient prior exposure to complete
encoding of one stimulus, they should look longer at a novel stimulus
that offers new opportunities to learn, showing a novelty preference. In
contrast, when infants have only limited prior exposure or partially
encoded one stimulus, they might look at that same stimulus for longer
to learn more about it, showing a familiarity preference.

However, empirical studies that systematically quantify familiarity
preferences for visual stimuli tend to be older, have smaller sample
sizes, and limited or no data available, making them unsuitable for
evaluating the robustness of the phenomenon (e.g., Hunter, Ames, \&
Koopman, 1983; Rose, Gottfried, Melloy-Carminar, \& Bridger, 1982).
Furthermore, this theoretical framework does not include formal criteria
to judge the completeness of encoding, limiting the precision of
predictions for new experiments. The dynamics in this framework are
instead often invoked retroactively, to explain unexpected findings. For
example, Johnson et al. (2009) studied rule learning in 8- and 11-month
old infants, finding a novelty preference in 8-month olds in one
condition and a familiarity preference in 11-month olds in three others
(as well as four conditions with no significant differences). They
interpreted these differences post hoc as indicating some combination of
greater complexity for certain rules over others and faster encoding by
older children.

To move from post hoc interpretations towards predictive frameworks of
looking time experiments, computational models are beginning to play a
bigger role in the literature. Across the cognitive sciences,
computational models are playing an essential role in facilitating
theory-building and elucidating cognitive phenomena precisely (Guest \&
Martin, 2021; Smaldino, 2020). For infant looking, formal models of
learning have successfully predicted infants' habituation and subsequent
preferences for novel stimuli (`dishabituation'). However, in contrast
to Hunter \& Ames (1988)'s framework, these formal models often do not
predict that infants will show a familiarity preference when given
limited learning experience (Sirois \& Mareschal, 2002). In a recent
example of such a model, Cao, Raz, Saxe, \& Frank (2022) proposed that
habituation and novelty preference could be explained by a rational
learner that takes noisy perceptual samples to maximize information gain
(RANCH: Cao et al., 2022). This model accurately predicted adult looking
time patterns in a self-paced habituation paradigm, reproducing both
habituation and novelty preferences. However, RANCH also did not predict
familiarity preferences at any stage of encoding, because its learning
policy to maximize information gain would always prioritize learning
about a novel stimulus over a repeated familiar stimulus, just to
varying degrees.

By contrast, other models do seem to contain either indirect or direct
predictions of familiarity preference. Kidd, Piantadosi, \& Aslin (2012)
proposed the ``Goldilocks effect'' -- infants' tendency to focus on
things that are neither too simple nor too complex - as a formal account
of infant looking. In this work, a Dirichlet-multinomial model was used
to learn the relative probability of objects appearing in specific
locations and related to infants' lookaway probabilities. A U-shaped
linking function between the model output and attention fit the infants'
behavior best, and although this effect was documented in a paradigm
that is very different from a typical looking time paradigm -- infants
were looking at a stream of rapidly changing events and their lookaways
were recorded -- it has been proposed that the preference for
intermediately complex events may account for familiarity preferences
(CITE; Stahl TICS paper i think). More recent work that uses rational
information gathering agents to explain infant looking behaviors
directly predicts familiarity preferences (Karni, Mattar, Emberson, \&
Daw, 2022). This model is similar to RANCH in that its learning policy
considers information gain, but it also considers another source of
value (i.e.~Need: how frequently the information about each stimulus
will be used). These dual processes generate non-monotonic changes in
looking time, which predict both familiarity preferences and novelty
preferences.

To evaluate and compare the predictions of these different model types,
it is necessary to have quantitative estimates of habituation, novelty
preferences, and familiarity preferences in infants. Under what
circumstances do infants look longer at a stimulus, following limited
exposure and thus potentially partial encoding?

In this paper, we aim to offer a stronger empirical foundation for
understanding how the duration of exposure influences looking
preferences. We conducted a set of experiments with preschoolers and
infants designed to test the conditions under which a familiarity
preference could be elicited. For preschoolers, we adapted a self-paced
looking time paradigm that was previously used to capture habituation
and novelty preference in adults (Cao et al., 2022). For infants, we
developed a novel within-participants measurement paradigm. This set of
experiments allow us to examine the validity of the assumption that the
field of developmental psychology holds: familiarity preference would
arise when the learners have limited experience with stimuli. To
preview, while preschoolers and infants show both habituation and
novelty preferences in our paradigm, we found no evidence for a
familiarity preference for either preschoolers or infants.

\hypertarget{experiment-1}{%
\section{Experiment 1}\label{experiment-1}}

Hunter \& Ames (1988) posit that younger participants are more likely to
exhibit familiarity preferences for a given stimulus due to their
reduced encoding speed. There was empirical evidence showing that
younger infants would show familiarity preferences in tasks in which
older infants show novelty preferences (Cyr grammar study). This
age-related change in preference may explain the lack of familiarity
preference observed in adults {[}Cao et al. (2022); Gustafson{]}. It is
possible that adults can process so fast that even brief exposure is
sufficient for completing the stimuli encoding.

To test this age hypothesis, we decided to test young children on an
experimental paradigm that has captured habituation and novelty
preference in adults (Cao et al., 2022). According to Hunter and Ames
(1988), young children should be slower in processing information,
making them more likely to show a familiarity preference.

\hypertarget{methods}{%
\subsection{Methods}\label{methods}}

\hypertarget{participants}{%
\subsection{Participants}\label{participants}}

66 children completed a task modified from the adult self-paced looking
time studies reported in CITE. Following our pre-registration (LINK), 2
children were excluded from the analysis because their performance in
the attention-check task failed to meet the inclusion criteria
(answering 4 out of the 8 attention check questions correctly). We also
excluded trials with looking time that were three absolute deviations
away from the median in the log-transformed space across participants
(\emph{N} = 83; ). The final datasets includes 64 children in total
(3YO: N = 18; 4YO: N = 26; 5YO: N = 20). All participants were recruited
in a university-affiliated research preschool.

\hypertarget{stimuli}{%
\subsection{Stimuli}\label{stimuli}}

We used a subset of stimuli created for the adult self-paced looking
time studies. In the previous study, we created a set of animated
creatures using Spore (a game developed by Maxis in 2008). Half of the
creatures had high perceptual complexity, and half had low perceptual
complexity. We used the high perceptual complexity stimuli for the
current study.

\hypertarget{procedures}{%
\subsection{Procedures}\label{procedures}}

Children were tested individually in a test room by an experimenter. The
experimenter invited the child to ``meet some monster friends'' and then
familiarized the child with the laptop computer used to present the
experiment. Before the test, each child went through a practice phase
where they practiced pressing the space bar to move on to the next
trial. The child was instructed that they can press the key and move on
to meet more monster friends whenever they want.

On each trial, the child would see an animated creature appear on the
screen. The child can move on to the next trial by pressing the space
bar. Each block consisted of six trials. Usually, the same creature will
be shown repeatedly (the background stimulus), but each block could
contain either zero or one deviant trial. Deviant trials were trials
that present a different creature from the background stimulus. Deviant
trials appeared on the second, the fourth, or the sixth trial of the
block. Each child saw eight blocks in total.

At the offset of each block, a memory task was presented to ensure
children are appropriately attending to the task. The memory task was a
2-Alternative Forced Choice (2AFC) question, asking the children to
identify which of the two stimuli they have seen before. The pair of
stimuli contained one stimulus used as a background stimulus in the
preceding block and a novel stimulus that did not appear anywhere else
in the experiment.

\begin{CodeChunk}
\begin{figure*}[h]

{\centering \includegraphics{figs/experimental_design-1} 

}

\caption[Experimental design of preschooler and infant experiments]{Experimental design of preschooler and infant experiments. There were three main differences: 1) Preschoolers responded with button presses, infants through lookaways, 2) preschoolers saw background trials after deviants, whereas deviants always appeared at the end in the infant experiments and 3) in preschoolers all trials were self-paced, whereas in infants only the last trial was self-paced.}\label{fig:experimental_design}
\end{figure*}
\end{CodeChunk}

\hypertarget{results-and-discussion}{%
\subsection{Results and discussion}\label{results-and-discussion}}

Children included in the final dataset showed a high level of accuracy
(\emph{M} = 0.97; \emph{SD} = 0.08) in responding to the memory task
question. This suggests that the children were engaged in the
experiment. We anticipated that the preschooler children would show
patterns of habituation and dishabituation similar to adults. We also
expected to see developmental changes in the shape of habituation
trajectories. Our pre-registered mixed-effect model included a three-way
interaction term between age (in months; scaled and centered), trial
number, and trial type (background or deviant) to predict
log-transformed looking time. The interaction between the trial number
and trial type was significant, suggesting the paradigm has captured
habituation and novelty preference in preschoolers (\(\beta\) = 0.14,
\emph{SE} = 0.02, \emph{t} = 6.22, \emph{p} \textless{} 0.01). However,
we did not find any significant interaction with age, nor was the main
effect significant (all \emph{p} \textgreater{} 0.1). We also explored
the potential familiarity preference by comparing the looking time at
the second background trial and the second deviant trial. Under the
Hunter \& Ames (1988) framework, the second trial in each block may be
most likely to yield a familiarity preference, since participants have
had the least amount of familiarization with the background stimulus in
a block. If there was a familiarity preference, participants should look
longer at a background trial than a deviant trial. However, we did not
find evidence supporting this prediction. We ran a mixed effect model
predicting looking time at the second trial with trial type as the
predictor. There was a significant trial type effect in the opposite
direction, suggesting participants looked longer at the deviant trial
than the background trial even with as little as one trial of
familiarization time (\(\beta\) = 0.41, \emph{SE} = 0.03, \emph{t} =
12.24, \emph{p} \textless{} 0.01).\\
In summary, this experiment captured habituation and novelty preferences
in preschoolers, replicating the patterns we saw in the previous adult
samples (CITE). However, under the current paradigm, we did not find any
evidence of a familiarity preferences. It is possible that processing in
preschoolers is already too fast for us to induce partial encoding in
this paradigm. In order to capture familiarity preference, we would need
to work with an even younger population (or a more complex stimulus).
However, the performance of 3-year-olds in this paradigm was noisier
than their older peers (Figure X). This trend suggested that the current
paradigm would not be suitable for testing even younger children. In
Experiment 2, we developed a new experimental paradigm to measure
habituation and looking preferences within participants in preverbal
infants.

\captionsetup{skip=4pt}

\begin{CodeChunk}
\begin{figure}[t!]

\includegraphics{figs/unnamed-chunk-13-1} \hfill{}

\caption[Looking times of preschoolers faceted by age showing habituation and dishabituation]{Looking times of preschoolers faceted by age showing habituation and dishabituation. Y-axis is log-transformed to reflect transformation of looking times in mixed effects models.}\label{fig:unnamed-chunk-13}
\end{figure}
\end{CodeChunk}

\hypertarget{experiment-2}{%
\section{Experiment 2}\label{experiment-2}}

\hypertarget{methods-1}{%
\subsection{Methods}\label{methods-1}}

In the infant paradigm, infants are familiarized to six unique stimuli
for different exposure durations within a single session in a blocked
design. This is in contrast to the standard infant
familiarization/habituation paradigm in which infants are familiarized
to only one stimulus throughout the experiment, which makes the effect
of exposure duration difficult to estimate. By presenting infants with
multiple blocks with varying familiarization times, we can directly
measure the effect of familiarization on looking time within
participants.

To get a dense estimate of exposure durations, we pre-registered and ran
two experiments, sequentially, with two sets of exposure durations. The
first experiment showed infants blocks of 0, 4 or 8 familiarizations
(Exp A; pre-registered
\href{https://osf.io/gux4f/?view_only=b4d6d0118dfa41a79fb431d389f4fecc}{here}).
The second experiment showed infants blocks of 1, 3 or 9
familiarizations (Exp B; pre-registered
\href{https://osf.io/w6pgu/?view_only=39ee108159884761a0c5bc68d11918df}{here}).

\hypertarget{participants-1}{%
\subsection{Participants}\label{participants-1}}

We tested a combined sample of 61 7-10 month old infants, with 29 in Exp
A and 32 in Exp B (\(M_{age}\) = 9.58 months, 29 female). 0 participants
were excluded completely due to fussiness. We also excluded an
additional individual test trials in which 1) infants looked at the
stimuli for less than a total 2 seconds, 2) there were momentary
external distractions in the home of the infant or 3) the gaze
classifier (see \textbf{Looking time coding}) had an average
classification confidence of less than 50\%. Data collection was
performed synchronously on Zoom, and infants were recruited from Lookit
(Scott \& Schulz, 2017) and Facebook.

\hypertarget{stimuli-1}{%
\subsection{Stimuli}\label{stimuli-1}}

Infants saw a different stimulus set from the preschoolers. In two
initial studies, not shown here, we showed infants the Spore stimulus
set used in preschoolers, in a slightly different experimental paradigm,
and failed to elicit replicable habituation, novelty or familiarity
preferences. In the current studies, we presented infants with a series
of animated animals, which we created using ``Quirky Animals'' assets
from Unity \href{https://tinyurl.com/469xxrn7}{link}. The animals were
walking, crawling or swimming, depending on the species.

\hypertarget{procedure}{%
\subsection{Procedure}\label{procedure}}

This experiment followed a block structure, where each block was divided
into two sections: 1) a familiarization period and 2) a test event. Each
block was preceded by our lab-standard attention getter, a salient
rotating star. During the familiarization period, the infant was
familiarized to a particular animal, the background, in a series of
familiarization trials. Each familiarization trial was a 5 second
sequence: curtains open for 1 second, the animated animal was presented
for 3 seconds, and then the curtains closed again for 1 second. We refer
to the number of times the curtain opened and closed as the
``familiarization duration'', which varied between blocks.

During the test event, the infant saw either the same background animal
again, or a novel animal, the deviant. The onset of the test event was
not marked by any visual markers, but a bell sound is played as the
curtains open, to maximize the chance of engagement during the test
trial. The test event used an infant-controlled procedure, in which the
experimenter terminated the trial when the infant looked away for more
than three consecutive seconds. Looking time was defined as the total
time that the infant spends looking at the screen from the onset of the
stimulus until the first two consecutive seconds of the infant looking
away from the screen. The discrepancy between the experimenter criterion
and the looking time criterion was to be conservative in stopping trials
to avoid early trial terminations. If the infant did not look away after
60 seconds of being presented with the test event, the next block
automatically began and infants' looking time for this test event was
recorded as 60 seconds.

Each baby saw six blocks: Three different familiarization durations (0,
4 and 8 in Exp. A, and 1, 3 and 9 in Exp. B) appeared twice each, once
for each test event type (background or deviant).

\hypertarget{looking-time-coding}{%
\subsection{Looking time coding}\label{looking-time-coding}}

To code the infants' gaze we used iCatcher+, a validated tool developed
for robust and automatic annotation of infants' gaze direction from
video (Erel, Potter, Jaffe-Dax, Lew-Williams, \& Bermano, 2022). To
obtain trial-wise looking times, we merged iCatcher+ annotations with
trial timing information, thereby fully replacing manual coding of
looking times.

\hypertarget{results-and-discussion-1}{%
\subsection{Results and discussion}\label{results-and-discussion-1}}

We pre-registered several linear mixed-effects models to test for
habituation, novelty preferences and familiarity preferences in our
paradigm. All models included a fixed effect of block number, and a
random effect of subject. To test the prediction that partial encoding
elicits familiarity preferences, while complete encoding elicits novelty
preferences, we pre-registered a model which allows for a non-linear
interaction between exposure duration by adding a quadratic effect of
familiarization duration, and its interaction with novelty. We found
that neither the main effect, nor the interaction of that quadratic term
were significant (main effect: \(\beta\) = 0.63; \emph{SE} = 0.89;
\emph{t} = 0.71; \emph{p} = 0.48; interaction: \(\beta\) = 0.75;
\emph{SE} = 1.63; \emph{t} = 0.46; \emph{p} = 0.65), while the
interaction of novelty with the linear term was significant (\(\beta\) =
4.32; \emph{SE} = 1.6; \emph{t} = 2.71; \emph{p} = 0.01). This suggests
that looking at the deviant increased as a function familiarization
duration, but that there is no special effect of partial encoding as
posited by H\&A. Furthermore, there was a significant decrease in
looking times to the familiar items as a function of familiarization
duration, indicating that infants habituated to familiar stimuli in our
paradigm (\(\beta\) = -2.66; \emph{SE} = 0.9; \emph{t} = -2.97; \emph{p}
= 0). Novelty preferences (i.e.~longer looking times at the deviant than
the background) were robust after 8 (\(\beta\) = 0.5; \emph{SE} = 0.19;
\emph{t} = 2.7; \emph{p} = 0.01) and 9 familiarizations (\(\beta\) =
0.63; \emph{SE} = 0.16; \emph{t} = 4.08; \emph{p} \textless{} 0.01), as
well as in the combined dataset (\(\beta\) = 0.57; \emph{SE} = 0.14;
\emph{t} = 4.19; \emph{p} \textless{} 0.01). We next tested specifically
for the existence of familiarity preference in our dataset. Similar to
the preschooler experiment, we hypothesized that familiarity preferences
are most likely to emerge in test trials following short
familiarizations. However, we did not find a significant effect of
novelty on looking times after 1 (\(\beta\) = -0.05; \emph{SE} = 0.19;
\emph{t} = -0.24; \emph{p} = 0.81), 3 (\(\beta\) = 0.35; \emph{SE} =
0.2; \emph{t} = 1.72; \emph{p} = 0.1) or 4 familiarizations (\(\beta\) =
-0.19; \emph{SE} = 0.21; \emph{t} = -0.89; \emph{p} = 0.38). Even when
maximizing power by combining test events following all three
familiarizations, there was no evidence of a familiarty preferences
(\(\beta\) = 0.05; \emph{SE} = 0.12; \emph{t} = 0.44; \emph{p} = 0.66).
To address whether the youngest infants in our sample may show
familiarity preferences, we ran an exploratory analysis asking whether
age interacted with the effect of novelty in the individual or combined
short exposure trials and found no evidence of age playing a role (all
p's \textgreater{} 0.4).

\captionsetup{skip=4pt}

\begin{CodeChunk}
\begin{figure}[h]

\includegraphics{figs/infant_results-1} \hfill{}

\caption[Looking times to background and deviant test trials as a function of familiarization duration]{Looking times to background and deviant test trials as a function of familiarization duration. We find evidence of habituation and novelty prefences, but familiarity preferences. Y-axis is log-transformed to reflect transformation of looking times in mixed effects models. Grey data and dashed line show baseline looking times without familiarization.}\label{fig:infant_results}
\end{figure}
\end{CodeChunk}

\hypertarget{general-discussion}{%
\section{General discussion}\label{general-discussion}}

Overall, we developed a novel looking time paradigm for infants and
preschoolers which tests the relationship between exposure duration and
attentional preferences within-subjects. In this paradigm, we found
strong evidence for habituation and novelty preferences. In contrast,
despite prematurely interrupting familiarization to induce partial
encoding, we failed to find attentional preferences for familiar stimuli
in either preschoolers or infants, suggesting that partial exposure did
not lead to familiarity preferences. Both the presence of novelty
preferences and habituation and the absence of familiarity preferences
were consistent across age groups, suggesting developmental continuity
in our paradigm.

These results are consistent with the idea that, in the current setting,
the decision of how long to look at a stimulus can be understood in
light of a simple information gain model, like the one presented in Cao
et al. (2022), across the lifespan. In contrast, our evidence does not
support the idea that attention in development could be driven by a
``knowledge gap'' motivating infants and children to complete their
encoding of familiar stimuli. If such a gap existed, it was too short to
detect in our studies. The failure to find familiarity preferences
through targeted partial encoding should also serve as a caution to
interpretations of looking time results in which familiarity preferences
are identified post-hoc, as it shows that such preferences may not arise
generically. Some of these may instead be false positives.

Of course, not finding familiarity preferences in our settings does not
rule out their existence, in our paradigm or in general. First,
familiarity preferences may be more subtle than novelty preferences, so
that the statistical power that is needed to find familiarity
preferences is higher than that achieved in the current study. A current
large-scale study by the ManyBabies consortium (Kosie et al., 2023)
which aims to test the predictions made by H\&A may give insight into
this possibility. Second, evoking familiarity preferences may depend on
the presentation mode of stimuli: While in our studies participants see
one stimulus, familiar or novel, at a time, many studies reporting
familiarity preferences follow a preferential looking set-up in which
infants are presented with both familiar and novel stimuli
simultaneously, and their relative looking time at each is recorded. It
may be that familiarity preferences arise due to the recognition of a
familiar stimulus among other stimuli, in which case our current
paradigm would not be suited to detect them (though see Gustafsson,
Francoeur, Blanchette, \& Sirois, 2021). Third, prior work has
emphasized the role of affective processes in familiarity preferences.
The mere exposure effect, a widely documented phenomenon in social
psychology, suggested that brief exposure to certain stimuli would be
sufficient to provoke a more positive affect Montoya, Horton, Vevea,
Citkowicz, \& Lauber (2017) . Therefore, it is possible that many
documented familiarity preferences arise because of underlying positive
affect. Including measurements that more directly tap into liking, such
as reaching or pointing, and relating them to looking time, may help
isolate the contribution of affect component in familiarity preferences.
Finally, and most importantly perhaps, the learning context in which
participants find themselves likely plays a critical role in whether
they will exhibit familiarity preferences. This context-dependence is
reflected in meta-analyses investigating familiarity preferences in
different paradigms. For example, when tested on word segmentation,
infants show a persistent preference for familiar stimuli throughout the
first year (Bergmann \& Cristia, 2016). In contrast, when tested on
statistical learning of novel words, infants show a consistent
preference for novel stimuli, from 4-month- to 11 months of age (Black
\& Bergmann, 2017). These seemingly contradictory results highlight the
need for theories that formalize the relationship between the learning
problem faced by participants and their attentional preferences. Recent
computational work has begun to provide normative accounts of how
context affects optimal exploration. For example, in environments in
which past and present are correlated, a preference for familiar stimuli
may arise to prepare for future encounters, while in uncorrelated
environments, novelty preferences are optimal{[}(under some assumptions;
see Dubey \& Griffiths (2020){]}. Similarly, in a rational analysis of
attentional preferences, Cao et al. (2022) show that ideal learners
attempting to maximize their expected information gain consistently seek
novelty when trying to learn a single concept. But it is possible that
once the learning goal or constraints on learning change e.g.~by
attempting to learn hierarchical concepts or imposing switch costs on
learning new concepts, optimal information-seeking may include attending
to familiar stimuli. In conclusion, we find robust evidence for
habituation and novelty preferences in preschoolers and infants, while
failing to observe a non-linearity in information seeking behavior,
despite attempting to induce partial encoding through a novel looking
paradigm in which we manipulate exposure within-subjects. Our findings
suggest that familiarity preferences do not arise under all
circumstances of limited exposure to stimuli, and that post-hoc
rationalizations of familiarity preferences observed in infant looking
time data should be made with care. Instead, we argue for the need for
new theories and data to understand the requisite features of stimuli
and learning contexts that elicit familiarity preferences.

\hypertarget{references}{%
\section{References}\label{references}}

\setlength{\parindent}{-0.1in} 
\setlength{\leftskip}{0.125in}

\noindent

\hypertarget{refs}{}
\begin{CSLReferences}{1}{0}
\leavevmode\vadjust pre{\hypertarget{ref-aslin2007s}{}}%
Aslin, R. N. (2007). What's in a look? \emph{Developmental Science},
\emph{10}(1), 48--53.

\leavevmode\vadjust pre{\hypertarget{ref-bergmann2016development}{}}%
Bergmann, C., \& Cristia, A. (2016). Development of infants'
segmentation of words from native speech: A meta-analytic approach.
\emph{Developmental Science}, \emph{19}(6), 901--917.

\leavevmode\vadjust pre{\hypertarget{ref-black2017quantifying}{}}%
Black, A., \& Bergmann, C. (2017). Quantifying infants' statistical word
segmentation: A meta-analysis. In \emph{39th annual meeting of the
cognitive science society} (pp. 124--129). Cognitive Science Society.

\leavevmode\vadjust pre{\hypertarget{ref-cao2022habituation}{}}%
Cao, A., Raz, G., Saxe, R., \& Frank, M. C. (2022). Habituation reflects
optimal exploration over noisy perceptual samples. \emph{Topics in
Cognitive Science}.

\leavevmode\vadjust pre{\hypertarget{ref-dubey2020reconciling}{}}%
Dubey, R., \& Griffiths, T. L. (2020). Reconciling novelty and
complexity through a rational analysis of curiosity. \emph{Psychological
Review}, \emph{127}(3), 455.

\leavevmode\vadjust pre{\hypertarget{ref-erel2022icatcher}{}}%
Erel, Y., Potter, C. E., Jaffe-Dax, S., Lew-Williams, C., \& Bermano, A.
H. (2022). iCatcher: A neural network approach for automated coding of
young children's eye movements. \emph{Infancy}, \emph{27}(4), 765--779.

\leavevmode\vadjust pre{\hypertarget{ref-guest2021computational}{}}%
Guest, O., \& Martin, A. E. (2021). How computational modeling can force
theory building in psychological science. \emph{Perspectives on
Psychological Science}, \emph{16}(4), 789--802.

\leavevmode\vadjust pre{\hypertarget{ref-gustafsson2021visual}{}}%
Gustafsson, E., Francoeur, C., Blanchette, I., \& Sirois, S. (2021).
Visual exploration in adults: Habituation, mere exposure, or optimal
level of arousal? \emph{Learning \& Behavior}, 1--9.

\leavevmode\vadjust pre{\hypertarget{ref-haith1980rules}{}}%
Haith, M. M. (1980). \emph{Rules that babies look by: The organization
of newborn visual activity}. Lawrence Erlbaum Associates.

\leavevmode\vadjust pre{\hypertarget{ref-hunter1988multifactor}{}}%
Hunter, M. A., \& Ames, E. W. (1988). A multifactor model of infant
preferences for novel and familiar stimuli. \emph{Advances in Infancy
Research}.

\leavevmode\vadjust pre{\hypertarget{ref-hunter1983effects}{}}%
Hunter, M. A., Ames, E. W., \& Koopman, R. (1983). Effects of stimulus
complexity and familiarization time on infant preferences for novel and
familiar stimuli. \emph{Developmental Psychology}, \emph{19}(3), 338.

\leavevmode\vadjust pre{\hypertarget{ref-johnson2009abstract}{}}%
Johnson, S. P., Fernandes, K. J., Frank, M. C., Kirkham, N., Marcus, G.,
Rabagliati, H., \& Slemmer, J. A. (2009). Abstract rule learning for
visual sequences in 8-and 11-month-olds. \emph{Infancy}, \emph{14}(1),
2--18.

\leavevmode\vadjust pre{\hypertarget{ref-karni2022}{}}%
Karni, G., Mattar, M. G., Emberson, L., \& Daw, N. (2022). \emph{A
rational information gathering account of infant exploratory behavior.
{[}Poster presentation{]}. RLDM}.

\leavevmode\vadjust pre{\hypertarget{ref-kidd2012goldilocks}{}}%
Kidd, C., Piantadosi, S. T., \& Aslin, R. N. (2012). The goldilocks
effect: Human infants allocate attention to visual sequences that are
neither too simple nor too complex. \emph{PloS One}, \emph{7}(5),
e36399.

\leavevmode\vadjust pre{\hypertarget{ref-kosie2023manybabies}{}}%
Kosie, J., Zettersten, M., Abu-Zhaya, R., Amso, D., Babineau, M.,
Baumgartne, H., et al.others. (2023). ManyBabies 5: A large-scale
investigation of the proposed shift from familiarity preference to
novelty preference in infant looking time.

\leavevmode\vadjust pre{\hypertarget{ref-montoya2017re}{}}%
Montoya, R. M., Horton, R. S., Vevea, J. L., Citkowicz, M., \& Lauber,
E. A. (2017). A re-examination of the mere exposure effect: The
influence of repeated exposure on recognition, familiarity, and liking.
\emph{Psychological Bulletin}, \emph{143}(5), 459.

\leavevmode\vadjust pre{\hypertarget{ref-rose1982familiarity}{}}%
Rose, S. A., Gottfried, A. W., Melloy-Carminar, P., \& Bridger, W. H.
(1982). Familiarity and novelty preferences in infant recognition
memory: Implications for information processing. \emph{Developmental
Psychology}, \emph{18}(5), 704.

\leavevmode\vadjust pre{\hypertarget{ref-scott2017lookit}{}}%
Scott, K., \& Schulz, L. (2017). Lookit (part 1): A new online platform
for developmental research. \emph{Open Mind}, \emph{1}(1), 4--14.

\leavevmode\vadjust pre{\hypertarget{ref-sirois2002models}{}}%
Sirois, S., \& Mareschal, D. (2002). Models of habituation in infancy.
\emph{Trends in Cognitive Sciences}, \emph{6}(7), 293--298.

\leavevmode\vadjust pre{\hypertarget{ref-smaldino2020translate}{}}%
Smaldino, P. E. (2020). How to translate a verbal theory into a formal
model. \emph{Social Psychology}.

\leavevmode\vadjust pre{\hypertarget{ref-zajonc1968attitudinal}{}}%
Zajonc, R. B. (1968). Attitudinal effects of mere exposure.
\emph{Journal of Personality and Social Psychology}, \emph{9}(2p2), 1.

\end{CSLReferences}

\bibliographystyle{apacite}


\end{document}
