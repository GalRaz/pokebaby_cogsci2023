% Template for Cogsci submission with R Markdown

% Stuff changed from original Markdown PLOS Template
\documentclass[10pt, letterpaper]{article}

\usepackage{cogsci}
\usepackage{pslatex}
\usepackage{float}
\usepackage{caption}

% amsmath package, useful for mathematical formulas
\usepackage{amsmath}

% amssymb package, useful for mathematical symbols
\usepackage{amssymb}

% hyperref package, useful for hyperlinks
\usepackage{hyperref}

% graphicx package, useful for including eps and pdf graphics
% include graphics with the command \includegraphics
\usepackage{graphicx}

% Sweave(-like)
\usepackage{fancyvrb}
\DefineVerbatimEnvironment{Sinput}{Verbatim}{fontshape=sl}
\DefineVerbatimEnvironment{Soutput}{Verbatim}{}
\DefineVerbatimEnvironment{Scode}{Verbatim}{fontshape=sl}
\newenvironment{Schunk}{}{}
\DefineVerbatimEnvironment{Code}{Verbatim}{}
\DefineVerbatimEnvironment{CodeInput}{Verbatim}{fontshape=sl}
\DefineVerbatimEnvironment{CodeOutput}{Verbatim}{}
\newenvironment{CodeChunk}{}{}

% cite package, to clean up citations in the main text. Do not remove.
\usepackage{apacite}

% KM added 1/4/18 to allow control of blind submission
\cogscifinalcopy

\usepackage{color}

% Use doublespacing - comment out for single spacing
%\usepackage{setspace}
%\doublespacing


% % Text layout
% \topmargin 0.0cm
% \oddsidemargin 0.5cm
% \evensidemargin 0.5cm
% \textwidth 16cm
% \textheight 21cm

\title{Familiarity preference something something???}


\author{Gal Raz$^1$ (galraz@mit.edu), \bf{Anjie Cao$^2$  (anjiecao@stanford.edu)},\\ \bf{Michael C. Frank$^2$ (mcfrank@stanford.edu)},
 and \bf{Rebecca Saxe$^1$ (saxe@mit.edu)} \\
$^1$Department of Brain and Cognitive Sciences, MIT, $^2$Department of Psychology, Stanford University \\ }


\begin{document}

\maketitle

\begin{abstract}
haha

\textbf{Keywords:}
decision making; learning; bayesian modeling; cognitive development
\end{abstract}

\hypertarget{introduction}{%
\section{Introduction}\label{introduction}}

\hypertarget{experiment-1}{%
\section{Experiment 1}\label{experiment-1}}

\hypertarget{methods}{%
\subsection{Methods}\label{methods}}

\hypertarget{participants}{%
\subsection{Participants}\label{participants}}

66 children completed a task modified from the adult self-paced looking
time studies reported in CITE. Following our pre-registration (LINK), 2
children were excluded from the analysis because their performance in
the attention-check task failed to meet the inclusion criteria. We also
excluded trials with looking time that were three absolute deviations
away from the median in the log-transformed space across participants.
The final datasets includes 64 children in total (3YO: N = 18; 4YO: N =
26; 5YO: N = 20 ). All participants were recruited in a
university-affiliated research preschool.

\hypertarget{stimuli}{%
\subsection{Stimuli}\label{stimuli}}

We used a subset of stimuli created for the adult self-paced looking
time studies. In the previous study, we created a set of animated
creatures using Spore (a game developed by Maxis in 2008). Half of the
creatures had high perceptual complexity, and half had low perceptual
complexity. We used the high perceptual complexity stimuli for the
current study.

\hypertarget{procedures}{%
\subsection{Procedures}\label{procedures}}

Children were tested individually in a test room by an experimenter. The
experimenter invited the child to ``meet some monster friends'' and then
familiarized the child with the laptop computer used to present the
experiment. Before the test, each child went through a practice phase
where they practiced pressing the space bar to move on to the next
trial. The child was instructed that they can press the key and move on
to meet more monster friends whenever they want.

On each trial, the child would see an animated creature appear on the
screen. The child can move on to the next trial by pressing the space
bar. Each block consisted of six trials. Usually, the same creature will
be shown repeatedly (the background stimulus), but each block could
contain either zero or one deviant trial. Deviant trials were trials
that present a different creature from the background stimulus. Deviant
trials appeared on the second, the fourth, or the sixth trial of the
block. Each child saw eight blocks in total.

At the offset of each block, a memory task was presented to ensure
children are appropriately attending to the task. The memory task was a
2-Alternative Forced Choice (2AFC) question, asking the children to
identify which of the two stimuli they have seen before. The pair of
stimuli contained one stimulus used as a background stimulus in the
preceding block and a novel stimulus that did not appear anywhere else
in the experiment.

\hypertarget{results-and-discussion}{%
\subsection{Results and discussion}\label{results-and-discussion}}

We anticipated that the preschooler children would show patterns of
habituation and dishabituation similar to adults. We also expected to
see developmental changes in the shape of habituation trajectories. Our
pre-registered mixed-effect mod includes a three-way interaction term
between age (in months), trial number, and trial type (background or
deviant) to predict log-transformed looking time. We only found a main
effect of trial number, suggesting that participants look shorter at
latter trials (\(\beta\) = -0.1, \emph{SE} = 0.05, \emph{t} = -2.05,
\emph{p} = 0.04). Since there was no reliable age effect, we deviated
from the analysis plan and ran a mixed-effect model only including
two-way interaction between trial number and trial type. In this model,
all predictors were significant (all \emph{p}\textless{} 0.01),
suggesting our paradigms successfully captured habituation and
dishabituation in preschoolers.

We also explored the potential familiarity preference by comparing the
looking time at the second background trial and the second deviant
trial. Under the Hunter \& Ames (1988), the second trial in each block
is most likely to yield a familiarity preference, since participants
receive the least amount of familiarization with the background stimulus
in a block. If there is a familiarity preference, participants should
look longer at a background trial than a deviant trial. However, we did
not find evidence supporting this prediction. We ran a mixed effect
model predicting looking time at the second trial with trial type as the
predictor. There was a significant trial type effect in the opposite
direction, suggesting participants looked longer at the deviant trial
than the background trial even with as little as one trial of
familiarization time (\(\beta\) = 0.41, \emph{SE} = 0.03, \emph{t} =
12.24, \emph{p} \textless{} 0.01).

In summary, This current experiment replicated the finding in We
captured habituation and dishabituation with a developmental sample.
More importantly, under the current paradigm, we did not find any
evidence of familiarity preference for this younger age group. We moved
to the infant samples in the next experiment.

\hypertarget{experiment-2}{%
\section{Experiment 2}\label{experiment-2}}

\hypertarget{methods-1}{%
\subsection{Methods}\label{methods-1}}

We anticipated that preschoolers would show patterns of habituation and
dishabituation similar to adults. We also expected to see developmental
changes in the shape of habituation trajectories. Our pre-registered
mixed-effect mod includes a three-way interaction term between age (in
months), trial number, and trial type (background or deviant) to predict
looking time. We found an interaction between {[}XXXX{]}, suggesting
that we successfully captured the habituation and dishabituation.
However, we did not find evidence suggesting that age is associated with
looking time (STATS). We also explored the potential familiarity
preference by comparing the looking time at the second background trial
and the second deviant trial. We ran a mixed effect model predicting
looking time at the second trial, with the interaction term between age
and trial type as the predictor. We found no evidence for familiarity
preference across age groups (CITE).

This current experiment replicated the finding in CITE. We captured
habituation and dishabituation with a developmental sample. More
importantly, under the current paradigm, we did not find any evidence of
familiarity preference for this younger age group. We moved to the
infant samples in the next experiment.

\hypertarget{results-and-discussion-1}{%
\subsection{Results and discussion}\label{results-and-discussion-1}}

\hypertarget{general-discussion}{%
\section{General discussion}\label{general-discussion}}

\hypertarget{references}{%
\section{References}\label{references}}

\setlength{\parindent}{-0.1in} 
\setlength{\leftskip}{0.125in}

\noindent

\bibliographystyle{apacite}


\end{document}
